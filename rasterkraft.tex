\documentclass[12pt]{article}
\title{F29 - Atomic Force Microscopy - Lab Report}
\author{Santa Clause}
\date{Christmas 2042}
\usepackage{graphicx}
\usepackage{hyperref}
\usepackage{multicol}
\usepackage{here}
\usepackage{color}
\newcommand{\blauefarbe}{\color{blue}}%
\newcommand{\AnswerFarbe}{\color{magenta}}%
\newcommand{\TextFarbe}{\color{black}}%

\textwidth=170mm
\textheight=240mm
\hoffset= -20mm       
\voffset= -30mm

\begin{document}

\maketitle

\begin{abstract}
Since it's development in 1981 the importance of scanning probe microscopy (SPM) has risen greatly in importance in a broad range of fields.
This lab course experiment is designed to give an introduction into the function and use of the atomic force microscope (AFM) as a widely used example for microscopy on the nanometer scale via the scanning probe technique. Therefore, the PID feedback loop of the microscope will be calibrated and characterizations like tip resonance curves and force-distance curves and limitations of the AFM will be looked at. Additionally, several samples will be studied. All in all, this experiment allows for relatively independent study due to the comparatively simple method of AFM and serves as a good introduction to nanoscale microscopy. 

This experiment was carried out ... \\
\end{abstract}


\begin{multicols}{2}
\input{background_new}
%\input{setup_new}
%\newpage
\input{results_new}
\input{discussion}
\input{references}

\end{multicols}

\end{document}


